
% Default to the notebook output style

    


% Inherit from the specified cell style.




    
\documentclass[11pt]{article}

    
    
    \usepackage[T1]{fontenc}
    % Nicer default font (+ math font) than Computer Modern for most use cases
    \usepackage{mathpazo}

    % Basic figure setup, for now with no caption control since it's done
    % automatically by Pandoc (which extracts ![](path) syntax from Markdown).
    \usepackage{graphicx}
    % We will generate all images so they have a width \maxwidth. This means
    % that they will get their normal width if they fit onto the page, but
    % are scaled down if they would overflow the margins.
    \makeatletter
    \def\maxwidth{\ifdim\Gin@nat@width>\linewidth\linewidth
    \else\Gin@nat@width\fi}
    \makeatother
    \let\Oldincludegraphics\includegraphics
    % Set max figure width to be 80% of text width, for now hardcoded.
    \renewcommand{\includegraphics}[1]{\Oldincludegraphics[width=.8\maxwidth]{#1}}
    % Ensure that by default, figures have no caption (until we provide a
    % proper Figure object with a Caption API and a way to capture that
    % in the conversion process - todo).
    \usepackage{caption}
    \DeclareCaptionLabelFormat{nolabel}{}
    \captionsetup{labelformat=nolabel}

    \usepackage{adjustbox} % Used to constrain images to a maximum size 
    \usepackage{xcolor} % Allow colors to be defined
    \usepackage{enumerate} % Needed for markdown enumerations to work
    \usepackage{geometry} % Used to adjust the document margins
    \usepackage{amsmath} % Equations
    \usepackage{amssymb} % Equations
    \usepackage{textcomp} % defines textquotesingle
    % Hack from http://tex.stackexchange.com/a/47451/13684:
    \AtBeginDocument{%
        \def\PYZsq{\textquotesingle}% Upright quotes in Pygmentized code
    }
    \usepackage{upquote} % Upright quotes for verbatim code
    \usepackage{eurosym} % defines \euro
    \usepackage[mathletters]{ucs} % Extended unicode (utf-8) support
    \usepackage[utf8x]{inputenc} % Allow utf-8 characters in the tex document
    \usepackage{fancyvrb} % verbatim replacement that allows latex
    \usepackage{grffile} % extends the file name processing of package graphics 
                         % to support a larger range 
    % The hyperref package gives us a pdf with properly built
    % internal navigation ('pdf bookmarks' for the table of contents,
    % internal cross-reference links, web links for URLs, etc.)
    \usepackage{hyperref}
    \usepackage{longtable} % longtable support required by pandoc >1.10
    \usepackage{booktabs}  % table support for pandoc > 1.12.2
    \usepackage[inline]{enumitem} % IRkernel/repr support (it uses the enumerate* environment)
    \usepackage[normalem]{ulem} % ulem is needed to support strikethroughs (\sout)
                                % normalem makes italics be italics, not underlines
    

    
    
    % Colors for the hyperref package
    \definecolor{urlcolor}{rgb}{0,.145,.698}
    \definecolor{linkcolor}{rgb}{.71,0.21,0.01}
    \definecolor{citecolor}{rgb}{.12,.54,.11}

    % ANSI colors
    \definecolor{ansi-black}{HTML}{3E424D}
    \definecolor{ansi-black-intense}{HTML}{282C36}
    \definecolor{ansi-red}{HTML}{E75C58}
    \definecolor{ansi-red-intense}{HTML}{B22B31}
    \definecolor{ansi-green}{HTML}{00A250}
    \definecolor{ansi-green-intense}{HTML}{007427}
    \definecolor{ansi-yellow}{HTML}{DDB62B}
    \definecolor{ansi-yellow-intense}{HTML}{B27D12}
    \definecolor{ansi-blue}{HTML}{208FFB}
    \definecolor{ansi-blue-intense}{HTML}{0065CA}
    \definecolor{ansi-magenta}{HTML}{D160C4}
    \definecolor{ansi-magenta-intense}{HTML}{A03196}
    \definecolor{ansi-cyan}{HTML}{60C6C8}
    \definecolor{ansi-cyan-intense}{HTML}{258F8F}
    \definecolor{ansi-white}{HTML}{C5C1B4}
    \definecolor{ansi-white-intense}{HTML}{A1A6B2}

    % commands and environments needed by pandoc snippets
    % extracted from the output of `pandoc -s`
    \providecommand{\tightlist}{%
      \setlength{\itemsep}{0pt}\setlength{\parskip}{0pt}}
    \DefineVerbatimEnvironment{Highlighting}{Verbatim}{commandchars=\\\{\}}
    % Add ',fontsize=\small' for more characters per line
    \newenvironment{Shaded}{}{}
    \newcommand{\KeywordTok}[1]{\textcolor[rgb]{0.00,0.44,0.13}{\textbf{{#1}}}}
    \newcommand{\DataTypeTok}[1]{\textcolor[rgb]{0.56,0.13,0.00}{{#1}}}
    \newcommand{\DecValTok}[1]{\textcolor[rgb]{0.25,0.63,0.44}{{#1}}}
    \newcommand{\BaseNTok}[1]{\textcolor[rgb]{0.25,0.63,0.44}{{#1}}}
    \newcommand{\FloatTok}[1]{\textcolor[rgb]{0.25,0.63,0.44}{{#1}}}
    \newcommand{\CharTok}[1]{\textcolor[rgb]{0.25,0.44,0.63}{{#1}}}
    \newcommand{\StringTok}[1]{\textcolor[rgb]{0.25,0.44,0.63}{{#1}}}
    \newcommand{\CommentTok}[1]{\textcolor[rgb]{0.38,0.63,0.69}{\textit{{#1}}}}
    \newcommand{\OtherTok}[1]{\textcolor[rgb]{0.00,0.44,0.13}{{#1}}}
    \newcommand{\AlertTok}[1]{\textcolor[rgb]{1.00,0.00,0.00}{\textbf{{#1}}}}
    \newcommand{\FunctionTok}[1]{\textcolor[rgb]{0.02,0.16,0.49}{{#1}}}
    \newcommand{\RegionMarkerTok}[1]{{#1}}
    \newcommand{\ErrorTok}[1]{\textcolor[rgb]{1.00,0.00,0.00}{\textbf{{#1}}}}
    \newcommand{\NormalTok}[1]{{#1}}
    
    % Additional commands for more recent versions of Pandoc
    \newcommand{\ConstantTok}[1]{\textcolor[rgb]{0.53,0.00,0.00}{{#1}}}
    \newcommand{\SpecialCharTok}[1]{\textcolor[rgb]{0.25,0.44,0.63}{{#1}}}
    \newcommand{\VerbatimStringTok}[1]{\textcolor[rgb]{0.25,0.44,0.63}{{#1}}}
    \newcommand{\SpecialStringTok}[1]{\textcolor[rgb]{0.73,0.40,0.53}{{#1}}}
    \newcommand{\ImportTok}[1]{{#1}}
    \newcommand{\DocumentationTok}[1]{\textcolor[rgb]{0.73,0.13,0.13}{\textit{{#1}}}}
    \newcommand{\AnnotationTok}[1]{\textcolor[rgb]{0.38,0.63,0.69}{\textbf{\textit{{#1}}}}}
    \newcommand{\CommentVarTok}[1]{\textcolor[rgb]{0.38,0.63,0.69}{\textbf{\textit{{#1}}}}}
    \newcommand{\VariableTok}[1]{\textcolor[rgb]{0.10,0.09,0.49}{{#1}}}
    \newcommand{\ControlFlowTok}[1]{\textcolor[rgb]{0.00,0.44,0.13}{\textbf{{#1}}}}
    \newcommand{\OperatorTok}[1]{\textcolor[rgb]{0.40,0.40,0.40}{{#1}}}
    \newcommand{\BuiltInTok}[1]{{#1}}
    \newcommand{\ExtensionTok}[1]{{#1}}
    \newcommand{\PreprocessorTok}[1]{\textcolor[rgb]{0.74,0.48,0.00}{{#1}}}
    \newcommand{\AttributeTok}[1]{\textcolor[rgb]{0.49,0.56,0.16}{{#1}}}
    \newcommand{\InformationTok}[1]{\textcolor[rgb]{0.38,0.63,0.69}{\textbf{\textit{{#1}}}}}
    \newcommand{\WarningTok}[1]{\textcolor[rgb]{0.38,0.63,0.69}{\textbf{\textit{{#1}}}}}
    
    
    % Define a nice break command that doesn't care if a line doesn't already
    % exist.
    \def\br{\hspace*{\fill} \\* }
    % Math Jax compatability definitions
    \def\gt{>}
    \def\lt{<}
    % Document parameters
    \title{Example\_ML\_Application}
    
    
    

    % Pygments definitions
    
\makeatletter
\def\PY@reset{\let\PY@it=\relax \let\PY@bf=\relax%
    \let\PY@ul=\relax \let\PY@tc=\relax%
    \let\PY@bc=\relax \let\PY@ff=\relax}
\def\PY@tok#1{\csname PY@tok@#1\endcsname}
\def\PY@toks#1+{\ifx\relax#1\empty\else%
    \PY@tok{#1}\expandafter\PY@toks\fi}
\def\PY@do#1{\PY@bc{\PY@tc{\PY@ul{%
    \PY@it{\PY@bf{\PY@ff{#1}}}}}}}
\def\PY#1#2{\PY@reset\PY@toks#1+\relax+\PY@do{#2}}

\expandafter\def\csname PY@tok@w\endcsname{\def\PY@tc##1{\textcolor[rgb]{0.73,0.73,0.73}{##1}}}
\expandafter\def\csname PY@tok@c\endcsname{\let\PY@it=\textit\def\PY@tc##1{\textcolor[rgb]{0.25,0.50,0.50}{##1}}}
\expandafter\def\csname PY@tok@cp\endcsname{\def\PY@tc##1{\textcolor[rgb]{0.74,0.48,0.00}{##1}}}
\expandafter\def\csname PY@tok@k\endcsname{\let\PY@bf=\textbf\def\PY@tc##1{\textcolor[rgb]{0.00,0.50,0.00}{##1}}}
\expandafter\def\csname PY@tok@kp\endcsname{\def\PY@tc##1{\textcolor[rgb]{0.00,0.50,0.00}{##1}}}
\expandafter\def\csname PY@tok@kt\endcsname{\def\PY@tc##1{\textcolor[rgb]{0.69,0.00,0.25}{##1}}}
\expandafter\def\csname PY@tok@o\endcsname{\def\PY@tc##1{\textcolor[rgb]{0.40,0.40,0.40}{##1}}}
\expandafter\def\csname PY@tok@ow\endcsname{\let\PY@bf=\textbf\def\PY@tc##1{\textcolor[rgb]{0.67,0.13,1.00}{##1}}}
\expandafter\def\csname PY@tok@nb\endcsname{\def\PY@tc##1{\textcolor[rgb]{0.00,0.50,0.00}{##1}}}
\expandafter\def\csname PY@tok@nf\endcsname{\def\PY@tc##1{\textcolor[rgb]{0.00,0.00,1.00}{##1}}}
\expandafter\def\csname PY@tok@nc\endcsname{\let\PY@bf=\textbf\def\PY@tc##1{\textcolor[rgb]{0.00,0.00,1.00}{##1}}}
\expandafter\def\csname PY@tok@nn\endcsname{\let\PY@bf=\textbf\def\PY@tc##1{\textcolor[rgb]{0.00,0.00,1.00}{##1}}}
\expandafter\def\csname PY@tok@ne\endcsname{\let\PY@bf=\textbf\def\PY@tc##1{\textcolor[rgb]{0.82,0.25,0.23}{##1}}}
\expandafter\def\csname PY@tok@nv\endcsname{\def\PY@tc##1{\textcolor[rgb]{0.10,0.09,0.49}{##1}}}
\expandafter\def\csname PY@tok@no\endcsname{\def\PY@tc##1{\textcolor[rgb]{0.53,0.00,0.00}{##1}}}
\expandafter\def\csname PY@tok@nl\endcsname{\def\PY@tc##1{\textcolor[rgb]{0.63,0.63,0.00}{##1}}}
\expandafter\def\csname PY@tok@ni\endcsname{\let\PY@bf=\textbf\def\PY@tc##1{\textcolor[rgb]{0.60,0.60,0.60}{##1}}}
\expandafter\def\csname PY@tok@na\endcsname{\def\PY@tc##1{\textcolor[rgb]{0.49,0.56,0.16}{##1}}}
\expandafter\def\csname PY@tok@nt\endcsname{\let\PY@bf=\textbf\def\PY@tc##1{\textcolor[rgb]{0.00,0.50,0.00}{##1}}}
\expandafter\def\csname PY@tok@nd\endcsname{\def\PY@tc##1{\textcolor[rgb]{0.67,0.13,1.00}{##1}}}
\expandafter\def\csname PY@tok@s\endcsname{\def\PY@tc##1{\textcolor[rgb]{0.73,0.13,0.13}{##1}}}
\expandafter\def\csname PY@tok@sd\endcsname{\let\PY@it=\textit\def\PY@tc##1{\textcolor[rgb]{0.73,0.13,0.13}{##1}}}
\expandafter\def\csname PY@tok@si\endcsname{\let\PY@bf=\textbf\def\PY@tc##1{\textcolor[rgb]{0.73,0.40,0.53}{##1}}}
\expandafter\def\csname PY@tok@se\endcsname{\let\PY@bf=\textbf\def\PY@tc##1{\textcolor[rgb]{0.73,0.40,0.13}{##1}}}
\expandafter\def\csname PY@tok@sr\endcsname{\def\PY@tc##1{\textcolor[rgb]{0.73,0.40,0.53}{##1}}}
\expandafter\def\csname PY@tok@ss\endcsname{\def\PY@tc##1{\textcolor[rgb]{0.10,0.09,0.49}{##1}}}
\expandafter\def\csname PY@tok@sx\endcsname{\def\PY@tc##1{\textcolor[rgb]{0.00,0.50,0.00}{##1}}}
\expandafter\def\csname PY@tok@m\endcsname{\def\PY@tc##1{\textcolor[rgb]{0.40,0.40,0.40}{##1}}}
\expandafter\def\csname PY@tok@gh\endcsname{\let\PY@bf=\textbf\def\PY@tc##1{\textcolor[rgb]{0.00,0.00,0.50}{##1}}}
\expandafter\def\csname PY@tok@gu\endcsname{\let\PY@bf=\textbf\def\PY@tc##1{\textcolor[rgb]{0.50,0.00,0.50}{##1}}}
\expandafter\def\csname PY@tok@gd\endcsname{\def\PY@tc##1{\textcolor[rgb]{0.63,0.00,0.00}{##1}}}
\expandafter\def\csname PY@tok@gi\endcsname{\def\PY@tc##1{\textcolor[rgb]{0.00,0.63,0.00}{##1}}}
\expandafter\def\csname PY@tok@gr\endcsname{\def\PY@tc##1{\textcolor[rgb]{1.00,0.00,0.00}{##1}}}
\expandafter\def\csname PY@tok@ge\endcsname{\let\PY@it=\textit}
\expandafter\def\csname PY@tok@gs\endcsname{\let\PY@bf=\textbf}
\expandafter\def\csname PY@tok@gp\endcsname{\let\PY@bf=\textbf\def\PY@tc##1{\textcolor[rgb]{0.00,0.00,0.50}{##1}}}
\expandafter\def\csname PY@tok@go\endcsname{\def\PY@tc##1{\textcolor[rgb]{0.53,0.53,0.53}{##1}}}
\expandafter\def\csname PY@tok@gt\endcsname{\def\PY@tc##1{\textcolor[rgb]{0.00,0.27,0.87}{##1}}}
\expandafter\def\csname PY@tok@err\endcsname{\def\PY@bc##1{\setlength{\fboxsep}{0pt}\fcolorbox[rgb]{1.00,0.00,0.00}{1,1,1}{\strut ##1}}}
\expandafter\def\csname PY@tok@kc\endcsname{\let\PY@bf=\textbf\def\PY@tc##1{\textcolor[rgb]{0.00,0.50,0.00}{##1}}}
\expandafter\def\csname PY@tok@kd\endcsname{\let\PY@bf=\textbf\def\PY@tc##1{\textcolor[rgb]{0.00,0.50,0.00}{##1}}}
\expandafter\def\csname PY@tok@kn\endcsname{\let\PY@bf=\textbf\def\PY@tc##1{\textcolor[rgb]{0.00,0.50,0.00}{##1}}}
\expandafter\def\csname PY@tok@kr\endcsname{\let\PY@bf=\textbf\def\PY@tc##1{\textcolor[rgb]{0.00,0.50,0.00}{##1}}}
\expandafter\def\csname PY@tok@bp\endcsname{\def\PY@tc##1{\textcolor[rgb]{0.00,0.50,0.00}{##1}}}
\expandafter\def\csname PY@tok@fm\endcsname{\def\PY@tc##1{\textcolor[rgb]{0.00,0.00,1.00}{##1}}}
\expandafter\def\csname PY@tok@vc\endcsname{\def\PY@tc##1{\textcolor[rgb]{0.10,0.09,0.49}{##1}}}
\expandafter\def\csname PY@tok@vg\endcsname{\def\PY@tc##1{\textcolor[rgb]{0.10,0.09,0.49}{##1}}}
\expandafter\def\csname PY@tok@vi\endcsname{\def\PY@tc##1{\textcolor[rgb]{0.10,0.09,0.49}{##1}}}
\expandafter\def\csname PY@tok@vm\endcsname{\def\PY@tc##1{\textcolor[rgb]{0.10,0.09,0.49}{##1}}}
\expandafter\def\csname PY@tok@sa\endcsname{\def\PY@tc##1{\textcolor[rgb]{0.73,0.13,0.13}{##1}}}
\expandafter\def\csname PY@tok@sb\endcsname{\def\PY@tc##1{\textcolor[rgb]{0.73,0.13,0.13}{##1}}}
\expandafter\def\csname PY@tok@sc\endcsname{\def\PY@tc##1{\textcolor[rgb]{0.73,0.13,0.13}{##1}}}
\expandafter\def\csname PY@tok@dl\endcsname{\def\PY@tc##1{\textcolor[rgb]{0.73,0.13,0.13}{##1}}}
\expandafter\def\csname PY@tok@s2\endcsname{\def\PY@tc##1{\textcolor[rgb]{0.73,0.13,0.13}{##1}}}
\expandafter\def\csname PY@tok@sh\endcsname{\def\PY@tc##1{\textcolor[rgb]{0.73,0.13,0.13}{##1}}}
\expandafter\def\csname PY@tok@s1\endcsname{\def\PY@tc##1{\textcolor[rgb]{0.73,0.13,0.13}{##1}}}
\expandafter\def\csname PY@tok@mb\endcsname{\def\PY@tc##1{\textcolor[rgb]{0.40,0.40,0.40}{##1}}}
\expandafter\def\csname PY@tok@mf\endcsname{\def\PY@tc##1{\textcolor[rgb]{0.40,0.40,0.40}{##1}}}
\expandafter\def\csname PY@tok@mh\endcsname{\def\PY@tc##1{\textcolor[rgb]{0.40,0.40,0.40}{##1}}}
\expandafter\def\csname PY@tok@mi\endcsname{\def\PY@tc##1{\textcolor[rgb]{0.40,0.40,0.40}{##1}}}
\expandafter\def\csname PY@tok@il\endcsname{\def\PY@tc##1{\textcolor[rgb]{0.40,0.40,0.40}{##1}}}
\expandafter\def\csname PY@tok@mo\endcsname{\def\PY@tc##1{\textcolor[rgb]{0.40,0.40,0.40}{##1}}}
\expandafter\def\csname PY@tok@ch\endcsname{\let\PY@it=\textit\def\PY@tc##1{\textcolor[rgb]{0.25,0.50,0.50}{##1}}}
\expandafter\def\csname PY@tok@cm\endcsname{\let\PY@it=\textit\def\PY@tc##1{\textcolor[rgb]{0.25,0.50,0.50}{##1}}}
\expandafter\def\csname PY@tok@cpf\endcsname{\let\PY@it=\textit\def\PY@tc##1{\textcolor[rgb]{0.25,0.50,0.50}{##1}}}
\expandafter\def\csname PY@tok@c1\endcsname{\let\PY@it=\textit\def\PY@tc##1{\textcolor[rgb]{0.25,0.50,0.50}{##1}}}
\expandafter\def\csname PY@tok@cs\endcsname{\let\PY@it=\textit\def\PY@tc##1{\textcolor[rgb]{0.25,0.50,0.50}{##1}}}

\def\PYZbs{\char`\\}
\def\PYZus{\char`\_}
\def\PYZob{\char`\{}
\def\PYZcb{\char`\}}
\def\PYZca{\char`\^}
\def\PYZam{\char`\&}
\def\PYZlt{\char`\<}
\def\PYZgt{\char`\>}
\def\PYZsh{\char`\#}
\def\PYZpc{\char`\%}
\def\PYZdl{\char`\$}
\def\PYZhy{\char`\-}
\def\PYZsq{\char`\'}
\def\PYZdq{\char`\"}
\def\PYZti{\char`\~}
% for compatibility with earlier versions
\def\PYZat{@}
\def\PYZlb{[}
\def\PYZrb{]}
\makeatother


    % Exact colors from NB
    \definecolor{incolor}{rgb}{0.0, 0.0, 0.5}
    \definecolor{outcolor}{rgb}{0.545, 0.0, 0.0}



    
    % Prevent overflowing lines due to hard-to-break entities
    \sloppy 
    % Setup hyperref package
    \hypersetup{
      breaklinks=true,  % so long urls are correctly broken across lines
      colorlinks=true,
      urlcolor=urlcolor,
      linkcolor=linkcolor,
      citecolor=citecolor,
      }
    % Slightly bigger margins than the latex defaults
    
    \geometry{verbose,tmargin=1in,bmargin=1in,lmargin=1in,rmargin=1in}
    
    

    \begin{document}
    
    
    \maketitle
    
    

    
    \section{Galaxy cluster mass estimates using machine learning
methods}\label{galaxy-cluster-mass-estimates-using-machine-learning-methods}

    This python notebook is meant to accompy Armitage et. al in prep. It
contains a few example code snippits to introduce the reader to the
python package sci-kit learn, and how to use sklearn to estimate the
mass of galaxy clusters. If you find this notebook useful feel free to
use parts of it in your own work and please cite Armitage et. al in
prep, thank you!

    \subsection{Setting up initial data}\label{setting-up-initial-data}

    First import all the packages we shall be using

    \begin{Verbatim}[commandchars=\\\{\}]
{\color{incolor}In [{\color{incolor}1}]:} \PY{c+c1}{\PYZsh{} Import standard packages}
        \PY{k+kn}{import} \PY{n+nn}{numpy} \PY{k}{as} \PY{n+nn}{np}
        \PY{k+kn}{import} \PY{n+nn}{scipy}\PY{n+nn}{.}\PY{n+nn}{stats} \PY{k}{as} \PY{n+nn}{sps}
        \PY{k+kn}{import} \PY{n+nn}{pandas} \PY{k}{as} \PY{n+nn}{pd}
        \PY{k+kn}{import} \PY{n+nn}{matplotlib}\PY{n+nn}{.}\PY{n+nn}{pyplot} \PY{k}{as} \PY{n+nn}{plt}
        \PY{k+kn}{from} \PY{n+nn}{matplotlib} \PY{k}{import} \PY{n}{rcParams}
        \PY{n}{rcParams}\PY{o}{.}\PY{n}{update}\PY{p}{(}\PY{p}{\PYZob{}}\PY{l+s+s1}{\PYZsq{}}\PY{l+s+s1}{text.latex.preamble}\PY{l+s+s1}{\PYZsq{}}\PY{p}{:}\PY{p}{[}\PY{l+s+sa}{r}\PY{l+s+s2}{\PYZdq{}}\PY{l+s+s2}{\PYZbs{}}\PY{l+s+s2}{usepackage}\PY{l+s+si}{\PYZob{}amsmath\PYZcb{}}\PY{l+s+s2}{\PYZdq{}}\PY{p}{]}\PY{p}{\PYZcb{}}\PY{p}{)}
        \PY{c+c1}{\PYZsh{} The example data file is pickled, not necessary for your own work}
        \PY{k+kn}{import} \PY{n+nn}{pickle}
        
        \PY{c+c1}{\PYZsh{} Import scikit learn}
        \PY{k+kn}{import} \PY{n+nn}{sklearn} \PY{k}{as} \PY{n+nn}{skl}
        \PY{k+kn}{from} \PY{n+nn}{sklearn}\PY{n+nn}{.}\PY{n+nn}{preprocessing} \PY{k}{import} \PY{n}{RobustScaler}
        \PY{k+kn}{from} \PY{n+nn}{sklearn}\PY{n+nn}{.}\PY{n+nn}{feature\PYZus{}selection} \PY{k}{import} \PY{n}{RFECV}
        \PY{k+kn}{from} \PY{n+nn}{sklearn}\PY{n+nn}{.}\PY{n+nn}{model\PYZus{}selection} \PY{k}{import} \PY{n}{KFold}
        \PY{k+kn}{from} \PY{n+nn}{sklearn}\PY{n+nn}{.}\PY{n+nn}{kernel\PYZus{}ridge} \PY{k}{import} \PY{n}{KernelRidge}
        \PY{k+kn}{from} \PY{n+nn}{sklearn}\PY{n+nn}{.}\PY{n+nn}{model\PYZus{}selection} \PY{k}{import} \PY{n}{GridSearchCV}
        \PY{k+kn}{from} \PY{n+nn}{sklearn}\PY{n+nn}{.}\PY{n+nn}{linear\PYZus{}model} \PY{k}{import} \PY{n}{Ridge}\PY{p}{,} \PY{n}{LinearRegression}
\end{Verbatim}


    Then read in the example data

    \begin{Verbatim}[commandchars=\\\{\}]
{\color{incolor}In [{\color{incolor}2}]:} \PY{k}{with} \PY{n+nb}{open}\PY{p}{(}\PY{l+s+s1}{\PYZsq{}}\PY{l+s+s1}{ExampleGalaxyData.dat}\PY{l+s+s1}{\PYZsq{}} \PY{p}{,} \PY{l+s+s1}{\PYZsq{}}\PY{l+s+s1}{rb}\PY{l+s+s1}{\PYZsq{}}\PY{p}{)} \PY{k}{as} \PY{n}{f}\PY{p}{:}
            \PY{n}{data}\PY{o}{=}\PY{n}{pickle}\PY{o}{.}\PY{n}{load}\PY{p}{(}\PY{n}{f}\PY{p}{)}
        \PY{n}{f}\PY{o}{.}\PY{n}{close}\PY{p}{(}\PY{p}{)}
\end{Verbatim}


    This is a pandas data frame containing a subset of the MACSIS clusters
(Barnes et al. 2017) at \(z=0.25\). The clusters are projected along a
cylinder of length \(10r_{200c}\), with an aperture of \(1r_{200c}\).
All galaxies within this cylinder above a stellar mass of
\(10^{10} \, \rm M_\odot\) are included and all clusters have at least
10 galaxies inside the cylinder.

The data is structured as follows:

    \begin{Verbatim}[commandchars=\\\{\}]
{\color{incolor}In [{\color{incolor}3}]:} \PY{n}{data}\PY{o}{.}\PY{n}{head}\PY{p}{(}\PY{l+m+mi}{2}\PY{p}{)}
\end{Verbatim}


\begin{Verbatim}[commandchars=\\\{\}]
{\color{outcolor}Out[{\color{outcolor}3}]:}               M200      R200  \textbackslash{}
        hm                             
        0000  4.746686e+15  3.264318   
        0001  1.654676e+15  2.297391   
        
                                                        posProj  \textbackslash{}
        hm                                                        
        0000  [[0.0, 0.0], [-0.104042094903, 0.971253299347]{\ldots}   
        0001  [[0.0, 0.0], [0.0873314232358, 1.42127895286],{\ldots}   
        
                                                        velProj  \textbackslash{}
        hm                                                        
        0000  [98.0720367432, 2734.51489258, 2234.1472168, -{\ldots}   
        0001  [724.227539062, 930.052185059, 2827.78588867, {\ldots}   
        
                                                       massProj  \textbackslash{}
        hm                                                        
        0000  [4.21088756951e+13, 657260405625.0, 1.15079706{\ldots}   
        0001  [1.25448113507e+13, 1.25876071092e+12, 5286234{\ldots}   
        
                                                       RposProj  
        hm                                                       
        0000  [0.0, 0.976809975894, 0.48081314504, 2.5351443{\ldots}  
        0001  [0.0, 1.42395949357, 1.55484570125, 1.24362182{\ldots}  
\end{Verbatim}
            
    M200 and R200 are the mass and radius of the cluster in \(\rm M_\odot\)
and \(\rm Mpc\) respectivly. The pos, vel, mass and Rpos 'proj' data are
arrays of the projected positions \(\rm [Mpc]\), velocities
\(\rm [km/s]\), stellar masses \(\rm [M_\odot]\) and projected radial
positions \(\rm [Mpc]\) of the galaxies in each cluster.

The index is the halo number and is a unique integer identifier of the
cluster.

    \subsection{Extracting features from the
data}\label{extracting-features-from-the-data}

The machine learning technqies used in this work need to be individual
floats (see
\href{https://dougalsutherland.github.io/skl-groups/index.html}{here}
for a different implimentation which takes the whole distribution as a
feature as used in Ntampaka et al. 2015).

Here we shall extract the mean, standard error on the mean, standard
deviation, skewness and kurtosis of the galaxy velocity, mass and
position distributions of each cluster as well as the number of
galaxies.

    \begin{Verbatim}[commandchars=\\\{\}]
{\color{incolor}In [{\color{incolor}4}]:} \PY{k}{def} \PY{n+nf}{FeatureCreation}\PY{p}{(}\PY{n}{df}\PY{p}{,}\PY{n}{spectroscopic}\PY{o}{=}\PY{k+kc}{True}\PY{p}{,}\PY{n}{Ngal}\PY{o}{=}\PY{k+kc}{True}\PY{p}{)}\PY{p}{:}
            \PY{c+c1}{\PYZsh{} The purpose of this function is the take the galaxy data of each MACSIS}
            \PY{c+c1}{\PYZsh{} cluster and create a set of features, which can then be fed into sklearn}
            \PY{c+c1}{\PYZsh{} regression algorithms}
            \PY{n}{featDf}\PY{o}{=}\PY{n}{pd}\PY{o}{.}\PY{n}{DataFrame}\PY{p}{(}\PY{p}{)}
        
            \PY{c+c1}{\PYZsh{} Features to extract}
            \PY{n}{fname}\PY{o}{=}\PY{p}{[}\PY{l+s+s1}{\PYZsq{}}\PY{l+s+s1}{kurt}\PY{l+s+s1}{\PYZsq{}}\PY{p}{,}\PY{l+s+s1}{\PYZsq{}}\PY{l+s+s1}{skew}\PY{l+s+s1}{\PYZsq{}}\PY{p}{,}\PY{l+s+s1}{\PYZsq{}}\PY{l+s+s1}{sem}\PY{l+s+s1}{\PYZsq{}}\PY{p}{,}\PY{l+s+s1}{\PYZsq{}}\PY{l+s+s1}{std}\PY{l+s+s1}{\PYZsq{}}\PY{p}{,}\PY{l+s+s1}{\PYZsq{}}\PY{l+s+s1}{mean}\PY{l+s+s1}{\PYZsq{}}\PY{p}{]}
            \PY{n}{spsFuncs}\PY{o}{=}\PY{p}{[}\PY{n}{sps}\PY{o}{.}\PY{n}{kurtosis}\PY{p}{,}\PY{n}{sps}\PY{o}{.}\PY{n}{skew}\PY{p}{,}\PY{n}{sps}\PY{o}{.}\PY{n}{sem}\PY{p}{,}\PY{n}{np}\PY{o}{.}\PY{n}{std}\PY{p}{,}\PY{n}{np}\PY{o}{.}\PY{n}{mean}\PY{p}{]}
                
            \PY{n}{abbrvNames}\PY{o}{=}\PY{p}{\PYZob{}}\PY{l+s+s1}{\PYZsq{}}\PY{l+s+s1}{RposProj}\PY{l+s+s1}{\PYZsq{}}\PY{p}{:}\PY{l+s+s1}{\PYZsq{}}\PY{l+s+s1}{Rpos}\PY{l+s+s1}{\PYZsq{}}\PY{p}{,}\PY{l+s+s1}{\PYZsq{}}\PY{l+s+s1}{massProj}\PY{l+s+s1}{\PYZsq{}}\PY{p}{:}\PY{l+s+s1}{\PYZsq{}}\PY{l+s+s1}{mass}\PY{l+s+s1}{\PYZsq{}}\PY{p}{\PYZcb{}}
        
            \PY{c+c1}{\PYZsh{} Add in velocity infomation if the data set is spectroscopic}
            \PY{k}{if} \PY{n}{spectroscopic}\PY{o}{==}\PY{k+kc}{True}\PY{p}{:}
                \PY{n}{abbrvNames}\PY{p}{[}\PY{l+s+s1}{\PYZsq{}}\PY{l+s+s1}{velProj}\PY{l+s+s1}{\PYZsq{}}\PY{p}{]}\PY{o}{=}\PY{l+s+s1}{\PYZsq{}}\PY{l+s+s1}{vel}\PY{l+s+s1}{\PYZsq{}}
        
            \PY{k}{for} \PY{n}{idx}\PY{p}{,}\PY{n}{func} \PY{o+ow}{in} \PY{n+nb}{enumerate}\PY{p}{(}\PY{n}{spsFuncs}\PY{p}{)}\PY{p}{:}
                \PY{n}{tmpDf}\PY{o}{=}\PY{n}{df}\PY{p}{[}\PY{l+s+s1}{\PYZsq{}}\PY{l+s+s1}{posProj}\PY{l+s+s1}{\PYZsq{}}\PY{p}{]}\PY{o}{.}\PY{n}{apply}\PY{p}{(}\PY{n}{func}\PY{p}{,}\PY{n}{args}\PY{o}{=}\PY{p}{(}\PY{l+m+mi}{0}\PY{p}{,}\PY{p}{)}\PY{p}{)}
                \PY{n}{tmpDf}\PY{o}{=}\PY{n}{tmpDf}\PY{o}{.}\PY{n}{apply}\PY{p}{(}\PY{n}{pd}\PY{o}{.}\PY{n}{Series}\PY{p}{)}
                \PY{n}{featDf}\PY{p}{[}\PY{l+s+s1}{\PYZsq{}}\PY{l+s+s1}{X\PYZus{}}\PY{l+s+s1}{\PYZsq{}}\PY{o}{+}\PY{n}{fname}\PY{p}{[}\PY{n}{idx}\PY{p}{]}\PY{p}{]}\PY{o}{=}\PY{n}{tmpDf}\PY{p}{[}\PY{l+m+mi}{0}\PY{p}{]}
                \PY{n}{featDf}\PY{p}{[}\PY{l+s+s1}{\PYZsq{}}\PY{l+s+s1}{Y\PYZus{}}\PY{l+s+s1}{\PYZsq{}}\PY{o}{+}\PY{n}{fname}\PY{p}{[}\PY{n}{idx}\PY{p}{]}\PY{p}{]}\PY{o}{=}\PY{n}{tmpDf}\PY{p}{[}\PY{l+m+mi}{1}\PY{p}{]}
        
            \PY{k}{for} \PY{n}{key} \PY{o+ow}{in} \PY{n}{abbrvNames}\PY{o}{.}\PY{n}{keys}\PY{p}{(}\PY{p}{)}\PY{p}{:}
                \PY{k}{for} \PY{n}{idx}\PY{p}{,}\PY{n}{func} \PY{o+ow}{in} \PY{n+nb}{enumerate}\PY{p}{(}\PY{n}{spsFuncs}\PY{p}{)}\PY{p}{:}
                    \PY{n}{tmpDf}\PY{o}{=}\PY{n}{df}\PY{p}{[}\PY{n}{key}\PY{p}{]}\PY{o}{.}\PY{n}{apply}\PY{p}{(}\PY{n}{func}\PY{p}{)}
                    \PY{c+c1}{\PYZsh{} Take the log of the velocity dispersion as it }
                    \PY{c+c1}{\PYZsh{} scales M \PYZbs{}propto \PYZbs{}sigma\PYZca{}3}
                    \PY{k}{if} \PY{p}{(}\PY{n}{abbrvNames}\PY{p}{[}\PY{n}{key}\PY{p}{]}\PY{o}{==}\PY{l+s+s1}{\PYZsq{}}\PY{l+s+s1}{vel}\PY{l+s+s1}{\PYZsq{}}\PY{p}{)}\PY{o}{\PYZam{}}\PY{p}{(}\PY{n}{fname}\PY{p}{[}\PY{n}{idx}\PY{p}{]}\PY{o}{==}\PY{l+s+s1}{\PYZsq{}}\PY{l+s+s1}{std}\PY{l+s+s1}{\PYZsq{}}\PY{p}{)}\PY{p}{:}
                        \PY{n}{featDf}\PY{p}{[}\PY{n}{abbrvNames}\PY{p}{[}\PY{n}{key}\PY{p}{]}\PY{o}{+}\PY{l+s+s1}{\PYZsq{}}\PY{l+s+s1}{\PYZus{}}\PY{l+s+s1}{\PYZsq{}}\PY{o}{+}\PY{n}{fname}\PY{p}{[}\PY{n}{idx}\PY{p}{]}\PY{p}{]}\PY{o}{=}\PY{n}{np}\PY{o}{.}\PY{n}{log10}\PY{p}{(}\PY{n}{tmpDf}\PY{p}{)}
                    \PY{k}{else}\PY{p}{:}
                        \PY{n}{featDf}\PY{p}{[}\PY{n}{abbrvNames}\PY{p}{[}\PY{n}{key}\PY{p}{]}\PY{o}{+}\PY{l+s+s1}{\PYZsq{}}\PY{l+s+s1}{\PYZus{}}\PY{l+s+s1}{\PYZsq{}}\PY{o}{+}\PY{n}{fname}\PY{p}{[}\PY{n}{idx}\PY{p}{]}\PY{p}{]}\PY{o}{=}\PY{n}{tmpDf}
        
            \PY{c+c1}{\PYZsh{} Log of the number of galaxies in the cluster}
            \PY{k}{if} \PY{n}{Ngal}\PY{o}{==}\PY{k+kc}{True}\PY{p}{:}
                \PY{n}{featDf}\PY{p}{[}\PY{l+s+s1}{\PYZsq{}}\PY{l+s+s1}{LogNgal}\PY{l+s+s1}{\PYZsq{}}\PY{p}{]}\PY{o}{=}\PY{n}{np}\PY{o}{.}\PY{n}{log10}\PY{p}{(}\PY{n}{df}\PY{p}{[}\PY{l+s+s1}{\PYZsq{}}\PY{l+s+s1}{massProj}\PY{l+s+s1}{\PYZsq{}}\PY{p}{]}\PY{o}{.}\PY{n}{apply}\PY{p}{(}\PY{n+nb}{len}\PY{p}{)}\PY{p}{)}
        
            \PY{k}{return} \PY{n}{featDf}
\end{Verbatim}


    \begin{Verbatim}[commandchars=\\\{\}]
{\color{incolor}In [{\color{incolor}5}]:} \PY{n}{featDf}\PY{o}{=}\PY{n}{FeatureCreation}\PY{p}{(}\PY{n}{data}\PY{p}{,}\PY{n}{spectroscopic}\PY{o}{=}\PY{k+kc}{True}\PY{p}{)}
        
        \PY{n+nb}{print}\PY{p}{(}\PY{l+s+s1}{\PYZsq{}}\PY{l+s+s1}{Number of features: }\PY{l+s+s1}{\PYZsq{}}\PY{p}{,}\PY{n+nb}{len}\PY{p}{(}\PY{n}{featDf}\PY{o}{.}\PY{n}{keys}\PY{p}{(}\PY{p}{)}\PY{p}{)}\PY{p}{)}
        \PY{n+nb}{print}\PY{p}{(}\PY{n}{featDf}\PY{o}{.}\PY{n}{keys}\PY{p}{(}\PY{p}{)}\PY{p}{)}
\end{Verbatim}


    \begin{Verbatim}[commandchars=\\\{\}]
Number of features:  26
Index(['X\_kurt', 'Y\_kurt', 'X\_skew', 'Y\_skew', 'X\_sem', 'Y\_sem', 'X\_std',
       'Y\_std', 'X\_mean', 'Y\_mean', 'Rpos\_kurt', 'Rpos\_skew', 'Rpos\_sem',
       'Rpos\_std', 'Rpos\_mean', 'mass\_kurt', 'mass\_skew', 'mass\_sem',
       'mass\_std', 'mass\_mean', 'vel\_kurt', 'vel\_skew', 'vel\_sem', 'vel\_std',
       'vel\_mean', 'LogNgal'],
      dtype='object')

    \end{Verbatim}

    Many machine learning algorithms assume that the features are centred
around 0 with a width of \(\mathcal{O}(1)\). We scale the data by
dividing by the 68\% spread and subtracting the median.

    \begin{Verbatim}[commandchars=\\\{\}]
{\color{incolor}In [{\color{incolor}6}]:} \PY{n}{X}\PY{o}{=}\PY{n}{featDf}\PY{o}{.}\PY{n}{values}
        \PY{n}{X} \PY{o}{=} \PY{n}{RobustScaler}\PY{p}{(}\PY{p}{)}\PY{o}{.}\PY{n}{fit\PYZus{}transform}\PY{p}{(}\PY{n}{X}\PY{p}{)}
\end{Verbatim}


    X now contains a N features for each cluster. We also need a feature
that we will train our model to reproduce. In this case it will simply
be the mass of the cluster logged, \(\log M_{200c}\).

    \begin{Verbatim}[commandchars=\\\{\}]
{\color{incolor}In [{\color{incolor}7}]:} \PY{n}{LogM200}\PY{o}{=}\PY{n}{np}\PY{o}{.}\PY{n}{log10}\PY{p}{(}\PY{n}{data}\PY{o}{.}\PY{n}{M200}\PY{o}{.}\PY{n}{values}\PY{p}{)}
\end{Verbatim}


    \subsection{Feature elimination}\label{feature-elimination}

Some of the features generated will contain little useful information
regarding \(M_{200c}\). We can use recursive feature elimination to get
rid unhelpful features.

    \begin{Verbatim}[commandchars=\\\{\}]
{\color{incolor}In [{\color{incolor}8}]:} \PY{k}{def} \PY{n+nf}{REF\PYZus{}Only}\PY{p}{(}\PY{n}{X}\PY{p}{,}\PY{n}{y}\PY{p}{,}\PY{n}{score}\PY{o}{=}\PY{l+s+s1}{\PYZsq{}}\PY{l+s+s1}{r2}\PY{l+s+s1}{\PYZsq{}}\PY{p}{)}\PY{p}{:}
            \PY{c+c1}{\PYZsh{} This estimator could be changed easily to give better results}
            \PY{c+c1}{\PYZsh{}estimator = SVR(kernel=\PYZdq{}linear\PYZdq{})}
            \PY{n}{estimator} \PY{o}{=} \PY{n}{LinearRegression}\PY{p}{(}\PY{p}{)}
            \PY{c+c1}{\PYZsh{}estimator = Ridge()}
        
            \PY{c+c1}{\PYZsh{} Possible regression scorers }
            \PY{c+c1}{\PYZsh{}[\PYZsq{}r2\PYZsq{},\PYZsq{}neg\PYZus{}median\PYZus{}absolute\PYZus{}error\PYZsq{},\PYZsq{}neg\PYZus{}mean\PYZus{}absolute\PYZus{}error\PYZsq{}]}
            
            \PY{n}{selector} \PY{o}{=} \PY{n}{RFECV}\PY{p}{(}\PY{n}{estimator}\PY{p}{,} \PY{n}{step}\PY{o}{=}\PY{l+m+mi}{1}\PY{p}{,} \PY{n}{cv}\PY{o}{=}\PY{l+m+mi}{5}\PY{p}{,}\PY{n}{scoring}\PY{o}{=}\PY{n}{score}\PY{p}{)}
        
            \PY{n}{selector} \PY{o}{=} \PY{n}{selector}\PY{o}{.}\PY{n}{fit}\PY{p}{(}\PY{n}{X}\PY{p}{,} \PY{n}{y}\PY{p}{)}
            \PY{c+c1}{\PYZsh{} X\PYZus{}new contains only the best features}
            \PY{n}{X\PYZus{}rfe} \PY{o}{=} \PY{n}{selector}\PY{o}{.}\PY{n}{fit\PYZus{}transform}\PY{p}{(}\PY{n}{X}\PY{p}{,} \PY{n}{y}\PY{p}{)}
        
            \PY{k}{return} \PY{n}{X\PYZus{}rfe}\PY{p}{,}\PY{n}{selector}\PY{o}{.}\PY{n}{ranking\PYZus{}}\PY{p}{,}\PY{n}{selector}\PY{o}{.}\PY{n}{grid\PYZus{}scores\PYZus{}}
\end{Verbatim}


    \begin{Verbatim}[commandchars=\\\{\}]
{\color{incolor}In [{\color{incolor}9}]:} \PY{n}{X\PYZus{}rfe}\PY{p}{,}\PY{n}{ranking}\PY{p}{,}\PY{n}{grid\PYZus{}scores}\PY{o}{=}\PY{n}{REF\PYZus{}Only}\PY{p}{(}\PY{n}{X}\PY{p}{,}\PY{n}{LogM200}\PY{p}{,}\PY{n}{score}\PY{o}{=}\PY{l+s+s1}{\PYZsq{}}\PY{l+s+s1}{r2}\PY{l+s+s1}{\PYZsq{}}\PY{p}{)}
        
        \PY{c+c1}{\PYZsh{}Plot the regressor score against number of features}
        \PY{n}{plt}\PY{o}{.}\PY{n}{plot}\PY{p}{(}\PY{n}{np}\PY{o}{.}\PY{n}{arange}\PY{p}{(}\PY{n+nb}{len}\PY{p}{(}\PY{n}{grid\PYZus{}scores}\PY{p}{)}\PY{p}{)}\PY{o}{+}\PY{l+m+mi}{1}\PY{p}{,}\PY{n}{grid\PYZus{}scores}\PY{p}{,}\PY{n}{lw}\PY{o}{=}\PY{l+m+mi}{2}\PY{p}{)}
        \PY{n}{plt}\PY{o}{.}\PY{n}{axvline}\PY{p}{(}\PY{n}{np}\PY{o}{.}\PY{n}{argmax}\PY{p}{(}\PY{n}{grid\PYZus{}scores}\PY{p}{)}\PY{o}{+}\PY{l+m+mi}{1}\PY{p}{,}\PY{n}{lw}\PY{o}{=}\PY{l+m+mi}{2}\PY{p}{,}\PY{n}{c}\PY{o}{=}\PY{l+s+s1}{\PYZsq{}}\PY{l+s+s1}{k}\PY{l+s+s1}{\PYZsq{}}\PY{p}{)}
        \PY{n}{plt}\PY{o}{.}\PY{n}{xlabel}\PY{p}{(}\PY{l+s+s1}{\PYZsq{}}\PY{l+s+s1}{\PYZdl{}N }\PY{l+s+s1}{\PYZbs{}}\PY{l+s+s1}{, }\PY{l+s+s1}{\PYZbs{}}\PY{l+s+s1}{mathrm}\PY{l+s+si}{\PYZob{}features\PYZcb{}}\PY{l+s+s1}{\PYZdl{}}\PY{l+s+s1}{\PYZsq{}}\PY{p}{,}\PY{n}{fontsize}\PY{o}{=}\PY{l+m+mi}{20}\PY{p}{)}
        \PY{n}{plt}\PY{o}{.}\PY{n}{ylabel}\PY{p}{(}\PY{l+s+s1}{\PYZsq{}}\PY{l+s+s1}{\PYZdl{}r\PYZca{}2\PYZdl{}}\PY{l+s+s1}{\PYZsq{}}\PY{p}{,}\PY{n}{fontsize}\PY{o}{=}\PY{l+m+mi}{20}\PY{p}{)}
        \PY{n}{plt}\PY{o}{.}\PY{n}{title}\PY{p}{(}\PY{l+s+s1}{\PYZsq{}}\PY{l+s+s1}{Recursive Feature Elimination}\PY{l+s+s1}{\PYZsq{}}\PY{p}{)}
        \PY{n}{plt}\PY{o}{.}\PY{n}{show}\PY{p}{(}\PY{p}{)}
\end{Verbatim}


    \begin{center}
    \adjustimage{max size={0.9\linewidth}{0.9\paperheight}}{output_19_0.png}
    \end{center}
    { \hspace*{\fill} \\}
    
    We have marked the maximum value of \(r^2\) on the plot above, the curve
is relativly flat beyond 10 features, but peaks at 16. These 16 features
are what is returned as X\_rfe.

    \subsection{Training the model}\label{training-the-model}

Finally we are ready to train the regression model. As per Armitage et
al. in prep we shall use ridge regression, though it is relativly simple
to replace with whatever model you desire in sklearn, you can change the
'model' keyword to 'OLR' or 'KRR' for ordinary linear regression and
kernal ridge regression respectivly. This function also tunes any
hyper-parameters using the GridSearchCV method.

    \begin{Verbatim}[commandchars=\\\{\}]
{\color{incolor}In [{\color{incolor}10}]:} \PY{k}{def} \PY{n+nf}{SplitAndTrain}\PY{p}{(}\PY{n}{X}\PY{p}{,}\PY{n}{y}\PY{p}{,}\PY{n}{splits}\PY{p}{,}\PY{n}{model}\PY{o}{=}\PY{l+s+s1}{\PYZsq{}}\PY{l+s+s1}{OLR}\PY{l+s+s1}{\PYZsq{}}\PY{p}{)}\PY{p}{:}
             \PY{c+c1}{\PYZsh{}y\PYZus{}pred is log\PYZus{}10 masses}
             \PY{n}{kf}\PY{o}{=}\PY{n}{KFold}\PY{p}{(}\PY{n}{n\PYZus{}splits}\PY{o}{=}\PY{n}{splits}\PY{p}{,} \PY{n}{random\PYZus{}state}\PY{o}{=}\PY{l+m+mi}{3}\PY{p}{,} \PY{n}{shuffle}\PY{o}{=}\PY{k+kc}{True}\PY{p}{)}
         
             \PY{n}{allRatios}\PY{o}{=}\PY{n}{np}\PY{o}{.}\PY{n}{array}\PY{p}{(}\PY{p}{[}\PY{p}{]}\PY{p}{)}
         
             \PY{c+c1}{\PYZsh{} Predected values of y}
             \PY{n}{y\PYZus{}pred\PYZus{}Arr}\PY{o}{=}\PY{n}{np}\PY{o}{.}\PY{n}{ones}\PY{p}{(}\PY{n+nb}{len}\PY{p}{(}\PY{n}{y}\PY{p}{)}\PY{p}{)}\PY{o}{+}\PY{n}{np}\PY{o}{.}\PY{n}{nan}
         
             \PY{c+c1}{\PYZsh{} Split the data into random \PYZsq{}splits\PYZsq{}. Shuffle round to always}
             \PY{c+c1}{\PYZsh{} test on one and train on the rest.}
             \PY{k}{for} \PY{n}{train\PYZus{}index}\PY{p}{,} \PY{n}{test\PYZus{}index} \PY{o+ow}{in} \PY{n}{kf}\PY{o}{.}\PY{n}{split}\PY{p}{(}\PY{n}{X}\PY{p}{)}\PY{p}{:}
                 
                 \PY{n}{X\PYZus{}train}\PY{p}{,} \PY{n}{X\PYZus{}test} \PY{o}{=} \PY{n}{X}\PY{p}{[}\PY{n}{train\PYZus{}index}\PY{p}{]}\PY{p}{,} \PY{n}{X}\PY{p}{[}\PY{n}{test\PYZus{}index}\PY{p}{]}
                 \PY{n}{y\PYZus{}train}\PY{p}{,} \PY{n}{y\PYZus{}test} \PY{o}{=} \PY{n}{y}\PY{p}{[}\PY{n}{train\PYZus{}index}\PY{p}{]}\PY{p}{,} \PY{n}{y}\PY{p}{[}\PY{n}{test\PYZus{}index}\PY{p}{]}
                 \PY{c+c1}{\PYZsh{} Now that I have my training and testing datasets train a }
                 \PY{c+c1}{\PYZsh{} given model.}
                 \PY{k}{if} \PY{n}{model}\PY{o}{==}\PY{l+s+s1}{\PYZsq{}}\PY{l+s+s1}{KRR}\PY{l+s+s1}{\PYZsq{}}\PY{p}{:}
                     \PY{n}{mod\PYZus{}clf} \PY{o}{=} \PY{n}{GridSearchCV}\PY{p}{(}
                                 \PY{n}{KernelRidge}\PY{p}{(}\PY{n}{kernel}\PY{o}{=}\PY{l+s+s1}{\PYZsq{}}\PY{l+s+s1}{rbf}\PY{l+s+s1}{\PYZsq{}}\PY{p}{,} \PY{n}{gamma}\PY{o}{=}\PY{l+m+mf}{0.1}\PY{p}{)}\PY{p}{,}
                                 \PY{n}{cv}\PY{o}{=}\PY{l+m+mi}{5}\PY{p}{,}
                                 \PY{n}{param\PYZus{}grid}\PY{o}{=}\PY{p}{\PYZob{}}
                                         \PY{l+s+s2}{\PYZdq{}}\PY{l+s+s2}{alpha}\PY{l+s+s2}{\PYZdq{}}\PY{p}{:} \PY{p}{[}\PY{l+m+mf}{1e0}\PY{p}{,} \PY{l+m+mf}{0.1}\PY{p}{,} \PY{l+m+mf}{1e\PYZhy{}2}\PY{p}{,} \PY{l+m+mf}{1e\PYZhy{}3}\PY{p}{]}\PY{p}{,}
                                         \PY{l+s+s2}{\PYZdq{}}\PY{l+s+s2}{gamma}\PY{l+s+s2}{\PYZdq{}}\PY{p}{:} \PY{n}{np}\PY{o}{.}\PY{n}{logspace}\PY{p}{(}\PY{o}{\PYZhy{}}\PY{l+m+mi}{2}\PY{p}{,} \PY{l+m+mi}{2}\PY{p}{,} \PY{l+m+mi}{5}\PY{p}{)}
                                         \PY{p}{\PYZcb{}}\PY{p}{)}
                     \PY{n}{y\PYZus{}pred} \PY{o}{=} \PY{n}{mod\PYZus{}clf}\PY{o}{.}\PY{n}{fit}\PY{p}{(}\PY{n}{X\PYZus{}train}\PY{p}{,} \PY{n}{y\PYZus{}train}\PY{p}{)}\PY{o}{.}\PY{n}{predict}\PY{p}{(}\PY{n}{X\PYZus{}test}\PY{p}{)}
         
                 \PY{k}{elif} \PY{n}{model}\PY{o}{==}\PY{l+s+s1}{\PYZsq{}}\PY{l+s+s1}{OLR}\PY{l+s+s1}{\PYZsq{}}\PY{p}{:}
                     \PY{n}{mod\PYZus{}clf} \PY{o}{=} \PY{n}{LinearRegression}\PY{p}{(}\PY{n}{fit\PYZus{}intercept}\PY{o}{=}\PY{k+kc}{True}\PY{p}{)}
                     \PY{n}{mod\PYZus{}clf}\PY{o}{.}\PY{n}{fit}\PY{p}{(}\PY{n}{X\PYZus{}train}\PY{p}{,} \PY{n}{y\PYZus{}train}\PY{p}{)}
                     \PY{n}{y\PYZus{}pred}\PY{o}{=}\PY{n}{mod\PYZus{}clf}\PY{o}{.}\PY{n}{predict}\PY{p}{(}\PY{n}{X\PYZus{}test}\PY{p}{)}
                     
         
                 \PY{k}{elif} \PY{n}{model}\PY{o}{==}\PY{l+s+s1}{\PYZsq{}}\PY{l+s+s1}{Ridge}\PY{l+s+s1}{\PYZsq{}}\PY{p}{:}
                     \PY{n}{mod\PYZus{}clf}\PY{o}{=}\PY{n}{GridSearchCV}\PY{p}{(}\PY{n}{Ridge}\PY{p}{(}\PY{n}{fit\PYZus{}intercept}\PY{o}{=}\PY{k+kc}{True}\PY{p}{)}\PY{p}{,} \PY{n}{cv}\PY{o}{=}\PY{l+m+mi}{5}\PY{p}{,}
                                \PY{n}{param\PYZus{}grid}\PY{o}{=}\PY{p}{\PYZob{}}\PY{l+s+s2}{\PYZdq{}}\PY{l+s+s2}{alpha}\PY{l+s+s2}{\PYZdq{}}\PY{p}{:} \PY{n}{np}\PY{o}{.}\PY{n}{linspace}\PY{p}{(}\PY{l+m+mf}{0.1}\PY{p}{,}\PY{l+m+mf}{9.1}\PY{p}{,}\PY{l+m+mi}{21}\PY{p}{)}\PY{p}{,}
                                            \PY{p}{\PYZcb{}}\PY{p}{)}
                     \PY{n}{mod\PYZus{}clf}\PY{o}{.}\PY{n}{fit}\PY{p}{(}\PY{n}{X\PYZus{}train}\PY{p}{,} \PY{n}{y\PYZus{}train}\PY{p}{)}
                     \PY{n}{y\PYZus{}pred}\PY{o}{=}\PY{n}{mod\PYZus{}clf}\PY{o}{.}\PY{n}{predict}\PY{p}{(}\PY{n}{X\PYZus{}test}\PY{p}{)}
         
                 \PY{n}{y\PYZus{}pred\PYZus{}Arr}\PY{p}{[}\PY{n}{test\PYZus{}index}\PY{p}{]}\PY{o}{=}\PY{n}{y\PYZus{}pred}
         
             \PY{k}{return} \PY{n}{mod\PYZus{}clf}\PY{p}{,}\PY{n}{y\PYZus{}pred\PYZus{}Arr}
\end{Verbatim}


    \begin{Verbatim}[commandchars=\\\{\}]
{\color{incolor}In [{\color{incolor}11}]:} \PY{n}{trainedModel}\PY{p}{,}\PY{n}{LogM200\PYZus{}pred}\PY{o}{=}\PY{n}{SplitAndTrain}\PY{p}{(}\PY{n}{X\PYZus{}rfe}\PY{p}{,}\PY{n}{LogM200}\PY{p}{,}\PY{n}{splits}\PY{o}{=}\PY{l+m+mi}{10}\PY{p}{,}\PY{n}{model}\PY{o}{=}\PY{l+s+s1}{\PYZsq{}}\PY{l+s+s1}{Ridge}\PY{l+s+s1}{\PYZsq{}}\PY{p}{)}
\end{Verbatim}


    \begin{Verbatim}[commandchars=\\\{\}]
{\color{incolor}In [{\color{incolor}12}]:} \PY{c+c1}{\PYZsh{} Plot the predicted mass vs true}
         
         \PY{n+nb}{print}\PY{p}{(}\PY{l+s+s1}{\PYZsq{}}\PY{l+s+s1}{Percentage scatter: }\PY{l+s+s1}{\PYZsq{}}\PY{p}{,}\PY{n+nb}{str}\PY{p}{(}\PY{l+m+mi}{100}\PY{o}{*}\PY{n}{np}\PY{o}{.}\PY{n}{std}\PY{p}{(}\PY{l+m+mi}{10}\PY{o}{*}\PY{o}{*}\PY{n}{LogM200\PYZus{}pred}\PY{o}{/}\PY{l+m+mi}{10}\PY{o}{*}\PY{o}{*}\PY{n}{LogM200}\PY{p}{)}\PY{p}{)}\PY{p}{[}\PY{p}{:}\PY{l+m+mi}{5}\PY{p}{]}\PY{p}{,} \PY{l+s+s1}{\PYZsq{}}\PY{l+s+s1}{\PYZpc{}}\PY{l+s+s1}{\PYZsq{}}\PY{p}{)}
         
         \PY{n}{plt}\PY{o}{.}\PY{n}{loglog}\PY{p}{(}\PY{l+m+mi}{10}\PY{o}{*}\PY{o}{*}\PY{n}{LogM200}\PY{p}{,}\PY{l+m+mi}{10}\PY{o}{*}\PY{o}{*}\PY{n}{LogM200\PYZus{}pred}\PY{p}{,}\PY{l+s+s1}{\PYZsq{}}\PY{l+s+s1}{o}\PY{l+s+s1}{\PYZsq{}}\PY{p}{)}
         
         \PY{n}{plt}\PY{o}{.}\PY{n}{xlim}\PY{p}{(}\PY{p}{[}\PY{l+m+mf}{3e14}\PY{p}{,}\PY{l+m+mf}{7e15}\PY{p}{]}\PY{p}{)}
         \PY{n}{plt}\PY{o}{.}\PY{n}{ylim}\PY{p}{(}\PY{p}{[}\PY{l+m+mf}{3e14}\PY{p}{,}\PY{l+m+mf}{7e15}\PY{p}{]}\PY{p}{)}
         \PY{n}{plt}\PY{o}{.}\PY{n}{plot}\PY{p}{(}\PY{p}{[}\PY{l+m+mf}{3e14}\PY{p}{,}\PY{l+m+mf}{1e16}\PY{p}{]}\PY{p}{,}\PY{p}{[}\PY{l+m+mf}{3e14}\PY{p}{,}\PY{l+m+mf}{1e16}\PY{p}{]}\PY{p}{,}\PY{l+s+s1}{\PYZsq{}}\PY{l+s+s1}{\PYZhy{}\PYZhy{}k}\PY{l+s+s1}{\PYZsq{}}\PY{p}{,}\PY{n}{lw}\PY{o}{=}\PY{l+m+mi}{2}\PY{p}{)}
         
         \PY{n}{plt}\PY{o}{.}\PY{n}{xlabel}\PY{p}{(}\PY{l+s+s1}{\PYZsq{}}\PY{l+s+s1}{\PYZdl{}M\PYZus{}}\PY{l+s+si}{\PYZob{}200c\PYZcb{}}\PY{l+s+s1}{ }\PY{l+s+s1}{\PYZbs{}}\PY{l+s+s1}{; }\PY{l+s+se}{\PYZbs{}\PYZbs{}}\PY{l+s+s1}{rm [M\PYZus{}}\PY{l+s+s1}{\PYZbs{}}\PY{l+s+s1}{odot]\PYZdl{}}\PY{l+s+s1}{\PYZsq{}}\PY{p}{,}\PY{n}{fontsize}\PY{o}{=}\PY{l+m+mi}{20}\PY{p}{)}
         \PY{n}{plt}\PY{o}{.}\PY{n}{ylabel}\PY{p}{(}\PY{l+s+s1}{\PYZsq{}}\PY{l+s+s1}{\PYZdl{}M\PYZus{}}\PY{l+s+s1}{\PYZob{}}\PY{l+s+s1}{200c, }\PY{l+s+se}{\PYZbs{}\PYZbs{}}\PY{l+s+s1}{rm pred\PYZcb{} }\PY{l+s+s1}{\PYZbs{}}\PY{l+s+s1}{; }\PY{l+s+se}{\PYZbs{}\PYZbs{}}\PY{l+s+s1}{rm [M\PYZus{}}\PY{l+s+s1}{\PYZbs{}}\PY{l+s+s1}{odot]\PYZdl{}}\PY{l+s+s1}{\PYZsq{}}\PY{p}{,}\PY{n}{fontsize}\PY{o}{=}\PY{l+m+mi}{20}\PY{p}{)}
         \PY{n}{plt}\PY{o}{.}\PY{n}{show}\PY{p}{(}\PY{p}{)}
\end{Verbatim}


    \begin{Verbatim}[commandchars=\\\{\}]
Percentage scatter:  7.781 \%

    \end{Verbatim}

    \begin{center}
    \adjustimage{max size={0.9\linewidth}{0.9\paperheight}}{output_24_1.png}
    \end{center}
    { \hspace*{\fill} \\}
    
    \subsubsection{Comparison to just using the velocity
dispersion}\label{comparison-to-just-using-the-velocity-dispersion}

This is just a quick comparison to show using only one feature, the
velocity dispersion, to train the model

    \begin{Verbatim}[commandchars=\\\{\}]
{\color{incolor}In [{\color{incolor}15}]:} \PY{n}{vel\PYZus{}stdOnly}\PY{o}{=}\PY{n}{featDf}\PY{p}{[}\PY{p}{[}\PY{l+s+s1}{\PYZsq{}}\PY{l+s+s1}{vel\PYZus{}std}\PY{l+s+s1}{\PYZsq{}}\PY{p}{]}\PY{p}{]}\PY{o}{.}\PY{n}{values}
         \PY{n}{vel\PYZus{}stdOnly} \PY{o}{=} \PY{n}{RobustScaler}\PY{p}{(}\PY{p}{)}\PY{o}{.}\PY{n}{fit\PYZus{}transform}\PY{p}{(}\PY{n}{vel\PYZus{}stdOnly}\PY{p}{)}
         \PY{n}{trainedModel\PYZus{}vstd}\PY{p}{,}\PY{n}{LogM200\PYZus{}vstd}\PY{o}{=}\PY{n}{SplitAndTrain}\PY{p}{(}\PY{n}{vel\PYZus{}stdOnly}\PY{p}{,}\PY{n}{LogM200}\PY{p}{,}\PY{n}{splits}\PY{o}{=}\PY{l+m+mi}{10}\PY{p}{,}\PY{n}{model}\PY{o}{=}\PY{l+s+s1}{\PYZsq{}}\PY{l+s+s1}{Ridge}\PY{l+s+s1}{\PYZsq{}}\PY{p}{)}
         
         \PY{n+nb}{print}\PY{p}{(}\PY{l+s+s1}{\PYZsq{}}\PY{l+s+s1}{Percentage scatter: }\PY{l+s+s1}{\PYZsq{}}\PY{p}{,}\PY{n+nb}{str}\PY{p}{(}\PY{l+m+mi}{100}\PY{o}{*}\PY{n}{np}\PY{o}{.}\PY{n}{std}\PY{p}{(}\PY{l+m+mi}{10}\PY{o}{*}\PY{o}{*}\PY{n}{LogM200\PYZus{}vstd}\PY{o}{/}\PY{l+m+mi}{10}\PY{o}{*}\PY{o}{*}\PY{n}{LogM200}\PY{p}{)}\PY{p}{)}\PY{p}{[}\PY{p}{:}\PY{l+m+mi}{5}\PY{p}{]}\PY{p}{,} \PY{l+s+s1}{\PYZsq{}}\PY{l+s+s1}{\PYZpc{}}\PY{l+s+s1}{\PYZsq{}}\PY{p}{)}
         \PY{n}{plt}\PY{o}{.}\PY{n}{loglog}\PY{p}{(}\PY{l+m+mi}{10}\PY{o}{*}\PY{o}{*}\PY{n}{LogM200}\PY{p}{,}\PY{l+m+mi}{10}\PY{o}{*}\PY{o}{*}\PY{n}{LogM200\PYZus{}vstd}\PY{p}{,}\PY{l+s+s1}{\PYZsq{}}\PY{l+s+s1}{o}\PY{l+s+s1}{\PYZsq{}}\PY{p}{)}
         \PY{n}{plt}\PY{o}{.}\PY{n}{xlim}\PY{p}{(}\PY{p}{[}\PY{l+m+mf}{3e14}\PY{p}{,}\PY{l+m+mf}{7e15}\PY{p}{]}\PY{p}{)}
         \PY{n}{plt}\PY{o}{.}\PY{n}{ylim}\PY{p}{(}\PY{p}{[}\PY{l+m+mf}{3e14}\PY{p}{,}\PY{l+m+mf}{7e15}\PY{p}{]}\PY{p}{)}
         \PY{n}{plt}\PY{o}{.}\PY{n}{plot}\PY{p}{(}\PY{p}{[}\PY{l+m+mf}{3e14}\PY{p}{,}\PY{l+m+mf}{1e16}\PY{p}{]}\PY{p}{,}\PY{p}{[}\PY{l+m+mf}{3e14}\PY{p}{,}\PY{l+m+mf}{1e16}\PY{p}{]}\PY{p}{,}\PY{l+s+s1}{\PYZsq{}}\PY{l+s+s1}{\PYZhy{}\PYZhy{}k}\PY{l+s+s1}{\PYZsq{}}\PY{p}{,}\PY{n}{lw}\PY{o}{=}\PY{l+m+mi}{2}\PY{p}{)}
         \PY{n}{plt}\PY{o}{.}\PY{n}{xlabel}\PY{p}{(}\PY{l+s+s1}{\PYZsq{}}\PY{l+s+s1}{\PYZdl{}M\PYZus{}}\PY{l+s+si}{\PYZob{}200c\PYZcb{}}\PY{l+s+s1}{ }\PY{l+s+s1}{\PYZbs{}}\PY{l+s+s1}{; }\PY{l+s+se}{\PYZbs{}\PYZbs{}}\PY{l+s+s1}{rm [M\PYZus{}}\PY{l+s+s1}{\PYZbs{}}\PY{l+s+s1}{odot]\PYZdl{}}\PY{l+s+s1}{\PYZsq{}}\PY{p}{,}\PY{n}{fontsize}\PY{o}{=}\PY{l+m+mi}{20}\PY{p}{)}
         \PY{n}{plt}\PY{o}{.}\PY{n}{ylabel}\PY{p}{(}\PY{l+s+s1}{\PYZsq{}}\PY{l+s+s1}{\PYZdl{}M\PYZus{}}\PY{l+s+s1}{\PYZob{}}\PY{l+s+s1}{200c, }\PY{l+s+se}{\PYZbs{}\PYZbs{}}\PY{l+s+s1}{rm pred, vstd\PYZcb{} }\PY{l+s+s1}{\PYZbs{}}\PY{l+s+s1}{; }\PY{l+s+se}{\PYZbs{}\PYZbs{}}\PY{l+s+s1}{rm [M\PYZus{}}\PY{l+s+s1}{\PYZbs{}}\PY{l+s+s1}{odot]\PYZdl{}}\PY{l+s+s1}{\PYZsq{}}\PY{p}{,}\PY{n}{fontsize}\PY{o}{=}\PY{l+m+mi}{20}\PY{p}{)}
         \PY{n}{plt}\PY{o}{.}\PY{n}{show}\PY{p}{(}\PY{p}{)}
\end{Verbatim}


    \begin{Verbatim}[commandchars=\\\{\}]
Percentage scatter:  33.80 \%

    \end{Verbatim}

    \begin{center}
    \adjustimage{max size={0.9\linewidth}{0.9\paperheight}}{output_26_1.png}
    \end{center}
    { \hspace*{\fill} \\}
    

    % Add a bibliography block to the postdoc
    
    
    
    \end{document}
